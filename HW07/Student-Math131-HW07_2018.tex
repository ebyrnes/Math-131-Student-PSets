\documentclass[12pt, fleqn]{article}
\usepackage{amsmath, amssymb, enumerate, mathpazo, mathrsfs, mathrsfs}
\usepackage[tmargin = 1.25in, lmargin = 1in, rmargin = 1in, bmargin = 1in]{geometry}
\usepackage{graphicx}
\usepackage{booktabs}
\usepackage{textcomp}

\newenvironment{problem}[1]{
\medskip \hrule \medskip
\noindent {\bf Problem #1.}
}{
\medskip \hrule \medskip
}

\newenvironment{solution}{
\medskip\noindent{\bf Solution.}}{}

\setlength{\headheight}{15.2pt}
\DeclareMathOperator{\var}{Var}
\DeclareMathOperator{\cov}{Cov}
\DeclareMathOperator{\bin}{Bin}
\DeclareMathOperator{\poi}{Poi}
\DeclareMathOperator{\expo}{Expo}
\DeclareMathOperator{\U}{U}
\DeclareMathOperator{\N}{N}

\newcommand{\thm}{\noindent{\bf Theorem.} }
\newcommand{\pf}{\noindent \emph{Proof.} }
\newcommand{\ex}{\noindent {\bf Example.} }
\newcommand{\Def}{\noindent {\bf Definition.} }
\newcommand{\note}{\noindent {\bf Note.} }
\newcommand{\question}{\noindent {\bf Question.} }
\newcommand{\cor}{\noindent {\bf Corollary.} }
\newcommand{\bart}[1]{\overline{{#1}}}
\newcommand{\inv}[1]{{#1}^{-1}}
\newcommand{\gen}[1]{\left<{#1}\right>}
\newcommand{\normg}{\trianglelefteq}

\newcommand{\partiald}[2]{\frac{\partial{#1}}{\partial{#2}}}
\newcommand{\nilrad}{\mathfrak{N}(R)}
\newcommand{\Jacobian}[4]{\left|
\begin{array}{cc}
\partiald{{#1}}{{#3}} & \partiald{{#1}}{{#4}} \\ \\
\partiald{{#2}}{{#3}} & \partiald{{#2}}{{#4}}
\end{array} \right|}
\newcommand{\determinant}[4]{\left|
\begin{array}{cc}
{#1} & {#2} \\
{#3} & {#4}
\end{array} \right|}




\begin{document}

%--------- Header ---------%
\pagestyle{myheadings}

%--------- Header ---------%
\begin{flushright}
Student Name \_\_\_\_\_\_\_\_\_\_\_\_\_\_\_\_\\
Math 131 \\
Homework 7 \\
Due Date: November 2, 2018 \\
\end{flushright}

\markright{%
 Math 131, Homework 7. 2018. \hfill \hfill  \hfill}


\begin{center} {\Large Homework 7}  \end{center}


\begin{problem}{1}
\\
Let $C$ be the Cantor set, as defined in 2.44 of Rudin.
\\
Recall that the Schroder-Berstein theorem (sometimes known as the Cantor-Bernstein- Schroder theorem) gives another way to show the equality of the cardinality of two sets. The theorem states: Given two sets $X$ and $Y$, $|X| = |Y|$ if and only if $|X| \leq |Y|$ and $|Y| \leq |X|$.

\begin{enumerate}

	\item Find a function $f : C \rightarrow [0, 1]$ that is onto. 
	\item Prove that Cantor set $C$ has the same cardinality as the closed interval $[0, 1] \in \mathbb{R}$ by using the function from part (a) and invoking the Schroder-Berstein theorem.

\end{enumerate}

\end{problem}

\begin{solution}

\end{solution}
\pagebreak
%~~~~~~~~~~~~~~~~~~~~~~~~~~~~~~~~~~~~~~~~~~~~~~~~~~~~~~~~~~~\\
\begin{problem}{2}
\\
For this problem, it may help to review exercise 2.9 in Rudin. Let $A$ be a subset of a metric space $X$ with closure $\overline{A}$. Let $A^\circ$ be the $interior$ of set $A$. That is, $A^\circ$ denotes the set of interior points of (recall definition 2.18(e) of Rudin). Let $\partial A$ be the $boundary$ of $A$. Thus,

$$A^\circ = \cup \{G \subset A : G \text{ is open} \}$$ $$\partial A = \overline{A} \backslash A^\circ$$

\begin{enumerate}

	\item Prove or disprove: If set $A$ is open, then it follows that $(\overline{A})^\circ = A$. 
	\item Prove or disprove: If set $A$ is connected, $\overline{A}$ will also be connected.
	\item Prove or disprove: If set $A$ is connected, $A^\circ$ will also be connected.

\end{enumerate}

\end{problem}

\begin{solution}

\end{solution}
\pagebreak
%~~~~~~~~~~~~~~~~~~~~~~~~~~~~~~~~~~~~~~~~~~~~~~~~~~~~~~~~~~~\\

\begin{problem}{3} 
\\
Convince yourself of the following theorem (do not turn in this part): If $\{G_{1}, G_{2}, G_{3}, \cdot \cdot \cdot \}$ is a countable collection of dense, open sets, then the intersection $\cap_{n=1}^{\infty}G_{n}$ is a dense (and therefore non-empty) subset of $\mathbb{R}$
\\
\\
Note that a set $G$ is dense in $\mathbb{R}$ if and only if $\overline{G} = \mathbb{R}$. So, for example, the set of rationals $\mathbb{Q}$ is dense in $\mathbb{R}$, but the set of integers $\mathbb{Z}$ is not.
\\
\\
We call a set $E$ in $\mathbb{R}$ nowhere-dense if $\overline{E}$ contains no nonempty open intervals. Note that $\mathbb{Z}$ is nowhere-dense in $\mathbb{R}$ since $\mathbb{Z}=\overline{\mathbb{Z}}$ contains no nonempty open intervals.

\begin{enumerate}

	\item Show that a set $E$ is nowhere-dense in $\mathbb{R}$ if and only if the complement of the closure, $(\overline{E})^{c}$, is dense in $\mathbb{R}$.
	\item Prove that the set $\mathbb{R}$ cannot be written as a countable union of nowhere dense sets. Hint: Use a proof by contradiction.
	
\end{enumerate}

\end{problem}
	
\begin{solution}


\end{solution}
\pagebreak

%~~~~~~~~~~~~~~~~~~~~~~~~~~~~~~~~~~~~~~~~~~~~~~~~~~~~~
\end{document}
