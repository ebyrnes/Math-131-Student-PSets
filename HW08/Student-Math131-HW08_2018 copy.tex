\documentclass[12pt, fleqn]{article}
\usepackage{amsmath, amssymb, enumerate, mathpazo, mathrsfs, mathrsfs}
\usepackage[tmargin = 1.25in, lmargin = 1in, rmargin = 1in, bmargin = 1in]{geometry}
\usepackage{graphicx}
\usepackage{booktabs}
\usepackage{textcomp}

\newenvironment{problem}[1]{
\medskip \hrule \medskip
\noindent {\bf Problem #1.}
}{
\medskip \hrule \medskip
}

\newenvironment{solution}{
\medskip\noindent{\bf Solution.}}{}

\setlength{\headheight}{15.2pt}
\DeclareMathOperator{\var}{Var}
\DeclareMathOperator{\cov}{Cov}
\DeclareMathOperator{\bin}{Bin}
\DeclareMathOperator{\poi}{Poi}
\DeclareMathOperator{\expo}{Expo}
\DeclareMathOperator{\U}{U}
\DeclareMathOperator{\N}{N}

\newcommand{\thm}{\noindent{\bf Theorem.} }
\newcommand{\pf}{\noindent \emph{Proof.} }
\newcommand{\ex}{\noindent {\bf Example.} }
\newcommand{\Def}{\noindent {\bf Definition.} }
\newcommand{\note}{\noindent {\bf Note.} }
\newcommand{\question}{\noindent {\bf Question.} }
\newcommand{\cor}{\noindent {\bf Corollary.} }
\newcommand{\bart}[1]{\overline{{#1}}}
\newcommand{\inv}[1]{{#1}^{-1}}
\newcommand{\gen}[1]{\left<{#1}\right>}
\newcommand{\normg}{\trianglelefteq}

\newcommand{\partiald}[2]{\frac{\partial{#1}}{\partial{#2}}}
\newcommand{\nilrad}{\mathfrak{N}(R)}
\newcommand{\Jacobian}[4]{\left|
\begin{array}{cc}
\partiald{{#1}}{{#3}} & \partiald{{#1}}{{#4}} \\ \\
\partiald{{#2}}{{#3}} & \partiald{{#2}}{{#4}}
\end{array} \right|}
\newcommand{\determinant}[4]{\left|
\begin{array}{cc}
{#1} & {#2} \\
{#3} & {#4}
\end{array} \right|}




\begin{document}

%--------- Header ---------%
\pagestyle{myheadings}

%--------- Header ---------%
\begin{flushright}
Student Name: \_\_\_\_\_\_\_\_\_\_\_\_\_\_\_\_\\
Math 131 \\
Homework 8 \\
Due Date: November 9, 2018 \\
\end{flushright}

\markright{%
 Math 131, Homework 8. 2018. \hfill \hfill  \hfill}


\begin{center} {\Large Homework 8}  \end{center}


\begin{problem}{1}

The terms $limit$ and $limit$ $point$ mean different things, but the notions are easy to confuse. For example, the constant sequence $1, 1, \cdot \cdot \cdot, 1,  \cdot \cdot \cdot$ is convergent with $limit$ 1. As a subset of the real line, however, its values are just equal to the set $\{ 1 \}$, which cannot have a limit point.

\begin{enumerate}

	\item Explain why the set $\{ 1 \}$ in $\mathbb{R}$ has no limit points.
	\item To clarify the notions of limit and limit point, prove the following statement: If a convergent sequence in a metric space has infinitely many distinct points, then its limit is a limit point of the set of points of the sequence.
	
\end{enumerate}

\end{problem}

\begin{solution}

\end{solution}
\pagebreak
%~~~~~~~~~~~~~~~~~~~~~~~~~~~~~~~~~~~~~~~~~~~~~~~~~~~~~~~~~~~\\
\begin{problem}{2}

Give an example of a sequence $\{ x_{n} \}$ with values in $[0, 1]$ that has the following property: For every $x \in [0,1]$, we can find a subsequence $\{ x_{n_{k}} \}$ such that $x_{n_{k}} \rightarrow x$ as $k \rightarrow \infty$.

\end{problem}

\begin{solution}

\end{solution}

\pagebreak
%~~~~~~~~~~~~~~~~~~~~~~~~~~~~~~~~~~~~~~~~~~~~~~~~~~~~~~~~~~~\\

\begin{problem}{3} 
\\
This exercise reveals additional interesting qualities of the Cantor set. 
\\ 
\\
Let $C$ be the Cantor set as defined in Rudin 2.44. Further, note that the sum $C + C = \{x+y : x,y \in C\}$. The goal of this exercise is to prove that $C + C = [0,2]$. Since $C \subset [0, 1]$, it follows immediately that $C + C \subset [0, 2]$. Thus, it remains only to show the inclusion in the other direction, namely that $[0, 2] \in \{x + y : x, y \in C \}$. That is, given any $s \in [0,2]$, we must show there exist numbers $x,y \in C$ that satisfy $x+y = s$. Do so through the following steps:

\begin{enumerate}

	\item Recall the notation from class in the construction of the Cantor set: $C = \cap_{n=0}^{\infty}K_{n}$ where $K_{n}$ is the union of $2^{N}$ closed subintervals of $[0,1]$ in which successive open middle third subintervals have been removed. Namely, $K_{0} = [0, 1]$, $K_{1} = [0, 1/3] \cup [2/3, 1]$, $K_{2} = [0, 1/9] \cup [2/9, 3/9] \cup [6/9, 7/9] \cup [8/9, 1]$, and so forth. Show that there exist $x_{1}, y_{1} \in K_{1}$ for which $x_{1} + y_{1} = s$.
	\item Use induction to show in general that for an arbitrary $n \in N$, we can always find $x_{n}, y_{n} \in K_{n}$ for which $x_{n} + y_{n} = s$.
	\item Show that we can find $x$ and $y$ in $C$ satisfying $x+  y = s$. (Note: The sequences $\{x_{n}\}$ and $\{y_{n} \}$ do not necessarily converge. Nonetheless, these can be used to produce the desired result.)

\end{enumerate}

\end{problem}
	
\begin{solution}


\end{solution}
\pagebreak

%~~~~~~~~~~~~~~~~~~~~~~~~~~~~~~~~~~~~~~~~~~~~~~~~~~~~~
\end{document}
