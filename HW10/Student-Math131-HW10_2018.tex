\documentclass[12pt, fleqn]{article}
\usepackage{amsmath, amssymb, enumerate, mathpazo, mathrsfs, mathrsfs}
\usepackage[tmargin = 1.25in, lmargin = 1in, rmargin = 1in, bmargin = 1in]{geometry}
\usepackage{graphicx}
\usepackage{booktabs}
\usepackage{textcomp}

\newenvironment{problem}[1]{
\medskip \hrule \medskip
\noindent {\bf Problem #1.}
}{
\medskip \hrule \medskip
}

\newenvironment{solution}{
\medskip\noindent{\bf Solution.}}{}

\setlength{\headheight}{15.2pt}
\DeclareMathOperator{\var}{Var}
\DeclareMathOperator{\cov}{Cov}
\DeclareMathOperator{\bin}{Bin}
\DeclareMathOperator{\poi}{Poi}
\DeclareMathOperator{\expo}{Expo}
\DeclareMathOperator{\U}{U}
\DeclareMathOperator{\N}{N}

\newcommand{\thm}{\noindent{\bf Theorem.} }
\newcommand{\pf}{\noindent \emph{Proof.} }
\newcommand{\ex}{\noindent {\bf Example.} }
\newcommand{\Def}{\noindent {\bf Definition.} }
\newcommand{\note}{\noindent {\bf Note.} }
\newcommand{\question}{\noindent {\bf Question.} }
\newcommand{\cor}{\noindent {\bf Corollary.} }
\newcommand{\bart}[1]{\overline{{#1}}}
\newcommand{\inv}[1]{{#1}^{-1}}
\newcommand{\gen}[1]{\left<{#1}\right>}
\newcommand{\normg}{\trianglelefteq}

\newcommand{\partiald}[2]{\frac{\partial{#1}}{\partial{#2}}}
\newcommand{\nilrad}{\mathfrak{N}(R)}
\newcommand{\Jacobian}[4]{\left|
\begin{array}{cc}
\partiald{{#1}}{{#3}} & \partiald{{#1}}{{#4}} \\ \\
\partiald{{#2}}{{#3}} & \partiald{{#2}}{{#4}}
\end{array} \right|}
\newcommand{\determinant}[4]{\left|
\begin{array}{cc}
{#1} & {#2} \\
{#3} & {#4}
\end{array} \right|}




\begin{document}

%--------- Header ---------%
\pagestyle{myheadings}

%--------- Header ---------%
\begin{flushright}
Student Name: \_\_\_\_\_\_\_\_\_\_\_\_\_\_\_\_\\
Math 131 \\
Homework 10 \\
Due Date: November 30, 2018 \\
\end{flushright}

\markright{%
 Math 131, Homework 10. 2018. \hfill \hfill  \hfill}


\begin{center} {\Large Homework 10}  \end{center}


\begin{problem}{1}

Let $a_{n} = \frac{n^{n}}{n!}$

\begin{enumerate}

	\item Prove that $\displaystyle{\lim_{n \to \infty}} \frac{a_{n+1}}{a_{n}} = e$
	\item Determine (with justification) $\displaystyle{\lim_{n \to \infty}} \frac{n}{(n!)^{\frac{1}{n}}}$
	
	Hint: There are some very helpful theorems in our textbook. Be sure to cite them when you use them.
\end{enumerate}

\end{problem}

\begin{solution}

\end{solution}
\pagebreak
%~~~~~~~~~~~~~~~~~~~~~~~~~~~~~~~~~~~~~~~~~~~~~~~~~~~~~~~~~~~\\
\begin{problem}{2}

Suppose $f(x) = x^{2}$. Is $f$ is uniformly continuous on $\mathbb{R}$? Justify your conclusion.

\end{problem}

\begin{solution}

\end{solution}

\pagebreak
%~~~~~~~~~~~~~~~~~~~~~~~~~~~~~~~~~~~~~~~~~~~~~~~~~~~~~~~~~~~\\

\begin{problem}{3} 

In $\mathbb{R}$, let $f$ be a continuous function on the closed interval $[0, 1]$ with range also contained in $[0, 1]$. Prove that $f$ must have a fixed point; that is, show $f(x) = x$ for at least one value of $x \in [0, 1]$. (Hint: use the Intermediate Value Theorem.)

\end{problem}
	
\begin{solution}

\end{solution}
\pagebreak

%~~~~~~~~~~~~~~~~~~~~~~~~~~~~~~~~~~~~~~~~~~~~~~~~~~~~~
\end{document}
